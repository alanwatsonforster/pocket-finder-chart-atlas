%!TEX root = main.tex

\chapter*{Preface}
\addcontentsline{toc}{chapter}{\protect\numberline{}Preface}

\begin{figure}[t]
\begin{center}
{\Large\bfseries Chart Legend}

\begin{tikzpicture}

\begin{scope}[xshift=8cm,yshift=0cm,yscale=-\chartscale]
\draw (0,0) -- (0,4);
\draw (0,0) -- (0.1,0) node [anchor=west] {$0^\circ$};
\draw (0,1) -- (0.1,1) node [anchor=west] {$1^\circ$};
\draw (0,2) -- (0.1,2) node [anchor=west] {$2^\circ$};
\draw (0,3) -- (0.1,3) node [anchor=west] {$3^\circ$};
\draw (0,4) -- (0.1,4) node [anchor=west] {$4^\circ$};
\end{scope}

\begin{scope}[xshift=0cm,yshift=-0cm,yscale=-1,xscale=1]
\draw (0,0) node [anchor=west] {Stars};
\drawstar{0}{0.5}{1.0} \draw (0.5,0.5) node [anchor=west] {1 mag};
\drawstar{0}{1.0}{2.0} \draw (0.5,1.0) node [anchor=west] {2 mag};
\drawstar{0}{1.5}{3.0} \draw (0.5,1.5) node [anchor=west] {3 mag};
\drawstar{0}{2.0}{4.0} \draw (0.5,2.0) node [anchor=west] {4 mag};
\drawstar{0}{2.5}{5.0} \draw (0.5,2.5) node [anchor=west] {5 mag};
\drawstar{0}{3.0}{6.0} \draw (0.5,3.0) node [anchor=west] {6 mag};
\drawstar{0}{3.5}{7.0} \draw (0.5,3.5) node [anchor=west] {7 mag};
\drawstar{0}{4.0}{8.0} \draw (0.5,4.0) node [anchor=west] {8 mag};
\drawstar{0}{4.5}{9.0} \draw (0.5,4.5) node [anchor=west] {9 mag};
\end{scope}

\begin{scope}[xshift=3cm,yshift=-0cm,yscale=-1]
\drawOC{0}{0.5}{0.15}\draw (0.5,0.5) node [anchor=west] {Open Cluster};
\drawGC{0}{1.0}{0.15}\draw (0.5,1.0) node [anchor=west] {Globular Cluster};
\drawSC{0}{1.5}{0.15}{0.15}{0}\draw (0.5,1.5) node [anchor=west] {Star Cloud or Asterism};
\drawBN{0}{2.0}{0.15}\draw (0.5,2.0) node [anchor=west] {Bright Nebula};
\drawPN{0}{2.5}{0.10}\draw (0.5,2.5) node [anchor=west] {Planetary Nebula};
\drawDN{0}{3.0}{0.15}\draw (0.5,3.0) node [anchor=west] {Dark Nebula};
\drawGAL{0}{3.5}{0.15}{0.10}{45}\draw (0.5,3.5) node [anchor=west] {Galaxy};
\end{scope}

\end{tikzpicture}

\end{center}
\end{figure}

This is an atlas of finder charts for the Messier, Caldwell, and Astronomical League Urban Observing Program objects. 

The charts have a field of 4 degrees at a scale of 13 mm per degree, show stars to magnitude 9.5, and represent of deep-sky objects following the current convention. Circles show fields with diameters of 1, 2, 3, and 4 degrees.

There are versions of the atlas with inverted-image and correct-image charts. 
\ifinverted
This version has inverted-image charts, with north up and east to the right, corresponding to the view through a Newtonian telescope or a refractor or Cassegrain telescope with a conventional diagonal.
\else
This version has correct-image charts, with north up and east to the left, corresponding to the view through a refractor or Cassegrain telescope either without a diagonal or with a correct-image diagonal.
\fi

These charts have one obvious flaw: all objects are drawn as either ellipses (galaxies and star clouds) or circles (everything else). This means, for example, that they do not correctly show the contours of M42 and M43.

I created this atlas to assist me as I observed under the light-polluted skies of Mexico City with a 70 mm f/6 refractor. I don't use a conventional finder with this telescope; instead, I star-hop using a 32 mm eyepiece with a true field of about 4 degrees. Under my usual observing conditions, very few objects are immediately obvious, and the charts tell me when I've successfully reached the desired field and where to focus my efforts.

Of course, the obvious question is, but why not just use use an all-sky atlas? I do indeed use the {\PSA} for star-hopping, but for confirming a field no all-sky atlas combines adequate depth with convenience at the telescope, and none really give me the sense of what I see through the eyepiece.

%Therefore, I created this atlas while guided by the following considerations. The charts should better approximate the view through the telescope at low power; they should be flipped, have a field of 4 degrees, show stars to 9.5 mag, and show the stars with smaller dots than typical. They should have a medium scale; I use 13 mm/deg throughout. They should complement the {\PSA}, my favored all-sky atlas, by adopting a similar graphic design and showing similar objects.  Finally, but no less importantly, they should be convenient to use at the telescope, with half-letter pages and a spiral binding that allows the atlas to fold back on itself, again to match the {\PSA}.

To some degree, the relation between these charts and the {\PSA} is similar to the relation between the large-scale and small-scale charts in \booktitle{The Observer’s Sky Atlas}. The small-scale all-sky atlas is to find the field and the large-scale charts are to confirm the field and locate the object. (I don’t use \booktitle{The Observer’s Sky Atlas} at the telescope, because the small-scale charts are too shallow and too narrow, the large-scale charts aren’t inverted, and the binding is inconvenient.) 

