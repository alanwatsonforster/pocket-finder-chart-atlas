%!TEX root = main.tex

\chapter*{Preface}
\addcontentsline{toc}{chapter}{\protect\numberline{}Preface}

This is an atlas of finder charts for the Messier, Caldwell, and Astronomical League Urban Program objects. For each object, it gives a chart with a field of 4 degrees, flipped east-west, and showing stars to magnitude 9.5. The chart style follows the current conventions for all-sky atlases.

I created this atlas to assist me as I observed under the light-polluted skies of Mexico City with a 70 mm f/6 wide-field refractor. I don't use a finder with this telescope; instead, I star-hop using a low-power 32 mm eyepiece with a true field of about 4 degrees. Once I’ve found the field, I switch to a higher-power eyepiece. Under my usual observing conditions, very few objects are immediately obvious, and the finder charts tell me when I've successfully star-hopped to the desired field and where to focus my efforts.

Of course, the obvious question is, but why not just use use an all-sky atlas? I do indeed use one for star-hopping, but for confirming a field none of them combine adequate depth (to magnitude 9.5) with convenience at the telescope (small size and a binding that folds flat or, better, back on itself), and none really give me the sense of what I see through the telescope (the images are not flipped and bright stars are represented as too big).

Therefore, I created this atlas while guided by the following considerations. The charts should better approximate the view through the telescope at low power; they should be flipped, have a field of 4 degrees, show stars to 9.5 mag, and show the stars with smaller dots than typical.
They should have a generous scale; I use 13 mm/deg throughout.
They should complement the {\PSA}, my favored all-sky atlas, by adopting a similar graphic design and showing similar objects.  
Finally, but no less importantly, they should be convenient to use at the telescope, with half-letter pages and a spiral binding that allows the atlas to fold back on itself, again to match the {\PSA}.

To some degree, the relation between these finder charts and the {\PSA} is similar to the relation between the large-scale and small-scale charts in \booktitle{The Observer’s Sky Atlas}. The small-scale all-sky atlas is to find the field and the large-scale finder charts are to confirm the field and locate the object. (I don’t use \booktitle{The Observer’s Sky Atlas} at the telescope, because the small-scale charts are too shallow and too narrow, the large-scale charts aren’t flipped, and the binding is inconvenient.) 

These finder charts have two known flaws. First, all objects are drawn as either ellipses (galaxies) or circles (everything else). This means, for example, that they do not correctly show the contours of M42 and M43. Second, they have no labels. To identify stars and neighboring objects, you will need to consult your all-sky atlas.
