%!TEX root = main.tex

\chapter{The Caldwell Objects}

The Caldwell objects were selected by Patrick Moore as a complement to the Messier objects. Again, I find O’Meara’s \emph{Deep-Sky Companion: The Caldwell Objects} to be excellent on the origin of the catalog, the appearance of the objects, and their nature.

I use C1 to C109 to label the Caldwell objects. This is common practice, but does not confirm to the IAU’s recommendation on nomenclature, as O’Meara rightly notes (pp.\ 14--15).

The following table lists the objects with their J2000 positions (decimal hours of right ascension and degrees of declination), the charts on which they appear in the {\PSA}, their types, and other names. I follow the corrections given by O’Meara (p.\ 15) with regards to C37, C49, C89, and C100.

For completeness, I include finder charts for all of the Caldwell objects, even bright ones like C41 (the Hyades).

I note that C41 (the Hyades) and C99 (the Coalsack Nebula) are much bigger than the finder charts. They are really binocular objects and are better located using a small-scale all-sky atlas.

% I note that page {\itshape xxviii} of the \emph{Pocket Sky Atlas} gives an index of the charts on which the Caldwell objects appear, along with their other names.

\clearpage

\begin{table}[p]
\setlength{\tabcolsep}{3pt}
\small
\begin{tabular}{lcclll}
\hline
Name&Position&PSA&Type&Other Names\\
\hline
C1   &$00.8$ $+85$&1&OC &NGC 188\\
C2   &$00.2$ $+73$&1/71&PN &NGC 40 = Bow-Tie Nebula\\
C3   &$12.3$ $+69$&31/41&Gal&NGC 4236\\
C4   &$21.0$ $+68$&61/71&BN &NGC 7023 = Iris Nebula\\
C5   &$03.8$ $+68$&1/11&Gal&IC 342 = Hidden Galaxy\\
C6   &$18.0$ $+67$&51/61&PN &NGC 6543 = Cat's Eye Nebula\\
C7   &$07.6$ $+66$&21&Gal&NGC 2403\\
C8   &$01.5$ $+63$&1/3&OC &NGC 559\\
C9   &$23.0$ $+63$&71/72&BN &Cave Nebula\\
C10  &$01.8$ $+61$&1/2&OC &NGC 663\\
C11  &$23.3$ $+61$&71/72&BN &NGC 7635 = Bubble Nebula\\
C12  &$20.6$ $+60$&61/62&Gal&NGC 6946 = Firecracker Galaxy\\
C13  &$01.3$ $+58$&1/3&OC &NGC 457 = Owl Cluster\\
C14  &$02.3$ $+57$&1/2&FC &NGC 869/884 = Double Cluster\\
C15  &$19.7$ $+51$&62&PN &NGC 6826 = Blinking Planetary\\
C16  &$22.3$ $+50$&73&OC &NGC 7243\\
C17  &$00.6$ $+49$&3&Gal&NGC 147\\
C18  &$00.6$ $+48$&3&Gal&NGC 185\\
C19  &$21.9$ $+47$&73&BN &IC 5146 = Cocoon Nebula\\
C20  &$21.0$ $+44$&62&BN &NGC 7000 = North America Nebula\\
C21  &$12.5$ $+44$&32/43&Gal&NGC 4449\\
C22  &$23.4$ $+43$&72&PN &NGC 7662 = Light Blue Snowball\\
C23  &$02.4$ $+42$&2&Gal&NGC 891\\
C24  &$03.3$ $+42$&13&Gal&NGC 1275 = Per A\\
C25  &$07.6$ $+39$&23&GC &NGC 2419\\
C26  &$12.3$ $+38$&32/43&Gal&NGC 4244\\
C27  &$20.2$ $+38$&62&BN &NGC 6888 = Crescent Nebula\\
C28  &$01.9$ $+38$&2&OC &NGC 752\\
C29  &$13.2$ $+37$&43&Gal&NGC 5005\\
C30  &$22.6$ $+34$&72/74&Gal&NGC 7331\\
C31  &$05.3$ $+34$&12&BN &IC 405 = Flaming Star Nebula\\
C32  &$12.7$ $+33$&43/45&Gal&NGC 4631 = Whale Galaxy\\
C33  &$20.9$ $+32$&62&BN &NGC 6992 = Eastern Veil Nebula\\
C34  &$20.8$ $+31$&62&BN &NGC 6960 = Western Veil Nebula\\
C35  &$13.0$ $+28$&43/45&Gal&NGC 4889\\
C36  &$12.6$ $+28$&43/45&Gal&NGC 4559\\
C37  &$20.2$ $+26$&62/64&OC &NGC 6885\\
C38  &$12.6$ $+26$&43/45&Gal&NGC 4565 = Needle Galaxy\\
C39  &$07.5$ $+21$&25&PN &NGC 2392 = Eskimo Nebula\\
C40  &$11.3$ $+18$&34&Gal&NGC 3626\\
\hline
\end{tabular}
\end{table}

\begin{table}[p]
\setlength{\tabcolsep}{3pt}
\small
\begin{tabular}{lcclll}
\hline
Name&Position&PSA&Type&Other Names\\
\hline
C41  &$04.4$ $+16$&15&OC &Mel 25 = Hyades\\
C42  &$21.0$ $+16$&64/75&GC &NGC 7006\\
C43  &$00.1$ $+16$&5/74&Gal&NGC 7814\\
C44  &$23.1$ $+12$&74&Gal&NGC 7479\\
C45  &$13.6$ $+09$&44&Gal&NGC 5248\\
C46  &$06.7$ $+09$&25/E&BN &NGC 2261 = Hubble's Variable Nebula\\
C47  &$20.6$ $+07$&64&GC &NGC 6934\\
C48  &$09.2$ $+07$&24/35&Gal&NGC 2775\\
C49  &$06.5$ $+05$&25/E&BN &NGC 2237/38/46 = Rosette Nebula\\
C50  &$06.5$ $+05$&25/E&OC &NGC 2244\\
C51  &$01.1$ $+02$&5/7&Gal&IC 1613\\
C52  &$12.8$ $-06$&47&Gal&NGC 4697\\
C53  &$10.1$ $-08$&37&Gal&NGC 3115 = The Spindle\\
C54  &$08.0$ $-11$&26&OC &NGC 2506\\
C55  &$21.1$ $-11$&77&PN &NGC 7009 = Saturn Nebula\\
C56  &$00.8$ $-12$&7&PN &NGC 246\\
C57  &$19.7$ $-15$&66&Gal&NGC 6822 = Barnard's Galaxy\\
C58  &$07.3$ $-16$&27&OC &NGC 2360 = Caroline's Cluster\\
C59  &$10.4$ $-19$&36/37&PN &NGC 3242 = Ghost of Jupiter\\
C60  &$12.0$ $-19$&36/47&Gal&NGC 4038 = NW Antennae Galaxy\\
C61  &$12.0$ $-19$&36/47&Gal&NGC 4039 = SE Antennae Galaxy\\
C62  &$00.8$ $-21$&7&Gal&NGC 247\\
C63  &$22.5$ $-21$&76/77&PN &NGC 7293 = Helix Nebula\\
C64  &$07.3$ $-25$&27&OC &NGC 2362 = $\tau$ CMa Cluster\\
C65  &$00.8$ $-25$&7/9&Gal&NGC 253 = Sculpter Galaxy\\
C66  &$14.7$ $-27$&46&GC &NGC 5694\\
C67  &$02.8$ $-30$&6/8&Gal&NGC 1097\\
C68  &$19.0$ $-37$&69&BN &NGC 6729 = R CrA Nebula\\
C69  &$17.2$ $-37$&58&PN &NGC 6302 = Bug Nebula\\
C70  &$00.9$ $-38$&9&Gal&NGC 300\\
C71  &$07.9$ $-39$&28&OC &NGC 2477\\
C72  &$00.2$ $-39$&9&Gal&NGC 55\\
C73  &$05.2$ $-40$&18&GC &NGC 1851\\
C74  &$10.1$ $-40$&39&PN &NGC 3132\\
C75  &$16.4$ $-41$&58&OC &NGC 6124\\
C76  &$16.9$ $-42$&58&OC &NGC 6231\\
C77  &$13.4$ $-43$&48/49&Gal&NGC 5128 = Cen A\\
C78  &$18.1$ $-44$&69&GC &NGC 6541\\
C79  &$10.3$ $-46$&39&GC &NGC 3201\\
C80  &$13.4$ $-47$&48/49&GC &NGC 5139 = $\omega$ Cen\\
\hline
\end{tabular}
\end{table}

\begin{table}[p]
\setlength{\tabcolsep}{3pt}
\small
\begin{tabular}{lcclll}
\hline
Name&Position&PSA&Type&Other Names\\
\hline
C81  &$17.4$ $-48$&58&GC &NGC 6352\\
C82  &$16.7$ $-49$&58&OC &NGC 6193\\
C83  &$13.1$ $-49$&49&Gal&NGC 4945\\
C84  &$13.8$ $-51$&48&GC &NGC 5286\\
C85  &$08.7$ $-53$&28&OC &IC 2391 = $\omicron$ Vel Cluster\\
C86  &$17.7$ $-54$&58&GC &NGC 6397\\
C87  &$03.2$ $-55$&19&GC &NGC 1261\\
C88  &$15.1$ $-56$&59&OC &NGC 5823\\
C89  &$16.3$ $-58$&58/60&OC &NGC 6087 = S Nor Cluster\\
C90  &$09.4$ $-58$&39&PN &NGC 2867\\
C91  &$11.1$ $-59$&38/40&OC &NGC 3532\\
C92  &$10.8$ $-60$&38/40&BN &NGC 3372 = $\eta$ Car Nebula\\
C93  &$19.2$ $-60$&69/70&GC &NGC 6752\\
C94  &$12.9$ $-60$&49/50&OC &NGC 4755 = Jewel Box\\
C95  &$16.1$ $-60$&59/60&OC &NGC 6025\\
C96  &$08.0$ $-61$&28/30&OC &NGC 2516 = Southern Beehive Cluster\\
C97  &$11.6$ $-62$&38/40&OC &NGC 3766\\
C98  &$12.7$ $-63$&49/50&OC &NGC 4609\\
C99  &$12.5$ $-64$&49/50&DN &Coalsack\\
C100 &$11.6$ $-63$&38/40&OC &Cr 249 = $\lambda$ Cen Cluster\\
C101 &$19.2$ $-64$&70&Gal&NGC 6744\\
C102 &$10.7$ $-64$&38/40&OC &IC 2602 = $\theta$ Car Cluster\\
C103 &$05.6$ $-69$&20/30/D&BN &NGC 2070 = Tarantula Nebula\\
C104 &$01.1$ $-71$&10/80&GC &NGC 362\\
C105 &$13.0$ $-71$&50&GC &NGC 4833\\
C106 &$00.4$ $-72$&10/80&GC &NGC 104 = 47 Tuc\\
C107 &$16.4$ $-72$&60&GC &NGC 6101\\
C108 &$12.4$ $-73$&40/50&GC &NGC 4372\\
C109 &$10.2$ $-81$&40&PN &NGC 3195\\
\hline
\end{tabular}
\end{table}

\clearpage

\twocolumn

\chart{C1}
\chart{C2}
\chart{C3}
\chart{C4}
\chart{C5}
\chart{C6}
\chart{C7}
\chart{C8}
\chart{C9}
\chart{C10}
\chart{C11}
\chart{C12}
\chart{C13}
\chart{C14}
\chart{C15}
\chart{C16}
\chart{C17}
\chart{C18}
\chart{C19}
\chart{C20}
\chart{C21}
\chart{C22}
\chart{C23}
\chart{C24}
\chart{C25}
\chart{C26}
\chart{C27}
\chart{C28}
\chart{C29}
\chart{C30}
\chart{C31}
\chart{C32}
\chart{C33}
\chart{C34}
\chart{C35}
\chart{C36}
\chart{C37}
\chart{C38}
\chart{C39}
\chart{C40}
\chart{C41}
\chart{C42}
\chart{C43}
\chart{C44}
\chart{C45}
\chart{C46}
\chart{C47}
\chart{C48}
\chart{C49}
\chart{C50}
\chart{C51}
\chart{C52}
\chart{C53}
\chart{C54}
\chart{C55}
\chart{C56}
\chart{C57}
\chart{C58}
\chart{C59}
\chart{C60}
\chart{C61}
\chart{C62}
\chart{C63}
\chart{C64}
\chart{C65}
\chart{C66}
\chart{C67}
\chart{C68}
\chart{C69}
\chart{C70}
\chart{C71}
\chart{C72}
\chart{C73}
\chart{C74}
\chart{C75}
\chart{C76}
\chart{C77}
\chart{C78}
\chart{C79}
\chart{C80}
\chart{C81}
\chart{C82}
\chart{C83}
\chart{C84}
\chart{C85}
\chart{C86}
\chart{C87}
\chart{C88}
\chart{C89}
\chart{C90}
\chart{C91}
\chart{C92}
\chart{C93}
\chart{C94}
\chart{C95}
\chart{C96}
\chart{C97}
\chart{C98}
\chart{C99}
\chart{C100}
\chart{C101}
\chart{C102}
\chart{C103}
\chart{C104}
\chart{C105}
\chart{C106}
\chart{C107}
\chart{C108}
\chart{C109}
\onecolumn
