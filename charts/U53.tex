\begin{tikzpicture}[xscale=\chartxscale,yscale=\chartyscale]
\drawframe{U53 = M4 = NGC 6121}{2.0000}
\drawGC{-0.0000}{+0.0000}{0.3000}
\drawradialscale{0.5000}
\drawradialscale{1.0000}
\drawradialscale{1.5000}
\drawradialscale{2.0000}
\drawstar{+1.4421}{+2.2941}{9.03}
\drawstar{+1.3206}{+2.3725}{9.03}
\drawstar{+1.6281}{+2.2330}{6.73}
\drawstar{+0.2862}{+2.1623}{9.47}
\drawstar{+0.7997}{+2.1574}{8.42}
\drawstar{+0.6734}{+2.3548}{4.67}
\drawstar{+0.9482}{+0.7615}{8.24}
\drawstar{+1.9220}{+1.9221}{8.46}
\drawstar{+1.5385}{+1.5989}{8.94}
\drawstar{+1.2904}{+1.6364}{9.39}
\drawstar{+1.5172}{+0.6532}{8.42}
\drawstar{+1.6319}{+1.4518}{8.43}
\drawstar{+1.7672}{+1.3915}{7.83}
\drawstar{+1.5493}{+1.2866}{8.92}
\drawstar{+2.6387}{+0.6126}{7.86}
\drawstar{+1.6245}{+1.5262}{7.95}
\drawstar{+1.9629}{+1.0326}{6.06}
\drawstar{+0.4022}{+0.5789}{9.43}
\drawstar{+0.3767}{+1.5344}{8.89}
\drawstar{+0.6247}{+1.1301}{8.86}
\drawstar{-0.1739}{+1.5005}{8.46}
\drawstar{+0.2222}{+2.1429}{8.69}
\drawstar{+0.7309}{+1.4088}{9.22}
\drawstar{-0.1887}{+0.8188}{9.18}
\drawstar{+0.7522}{+0.8780}{9.16}
\drawstar{-0.4136}{+2.0596}{8.03}
\drawstar{+0.5410}{+0.9320}{2.91}
\drawstar{-0.3158}{+1.1706}{8.67}
\drawstar{-1.8672}{+0.4996}{8.31}
\drawstar{-1.5755}{+1.0449}{8.71}
\drawstar{-1.2274}{+1.9822}{7.66}
\drawstar{-1.0972}{+1.7708}{7.56}
\drawstar{-1.4328}{+1.5379}{7.86}
\drawstar{-1.4986}{+1.4013}{4.76}
\drawstar{-1.4482}{+1.3694}{7.37}
\drawstar{-1.0527}{+1.0673}{6.99}
\drawstar{+2.1703}{+0.2154}{8.98}
\drawstar{+2.7247}{+0.0232}{9.11}
\drawstar{+2.5386}{+0.0186}{9.39}
\drawstar{+2.2463}{-0.9001}{8.45}
\drawstar{+2.6027}{-1.0568}{8.07}
\drawstar{+2.1654}{-1.2877}{9.34}
\drawstar{+2.7008}{-0.2426}{9.12}
\drawstar{+2.6796}{-0.4662}{9.04}
\drawstar{+2.6667}{-0.5835}{8.17}
\drawstar{+1.8385}{-0.6989}{7.61}
\drawstar{+1.1737}{-1.5218}{7.06}
\drawstar{+2.6757}{-0.6564}{6.96}
\drawstar{-0.0445}{+0.2550}{8.93}
\drawstar{-0.4937}{-1.2942}{9.28}
\drawstar{-0.6444}{-1.1648}{8.68}
\drawstar{-0.3736}{-1.1959}{9.01}
\drawstar{-0.7933}{+0.2657}{8.41}
\drawstar{+0.1124}{-1.3537}{8.63}
\drawstar{+0.4373}{-1.4058}{9.48}
\drawstar{-0.2425}{-0.2814}{8.43}
\drawstar{+0.5071}{-0.7067}{8.99}
\drawstar{-0.8052}{-0.6662}{8.30}
\drawstar{-0.3544}{-1.1420}{8.70}
\drawstar{-0.2045}{-0.6253}{7.50}
\drawstar{-0.5045}{-0.0435}{8.12}
\drawstar{-2.1255}{-1.6096}{8.31}
\drawstar{-2.4868}{-0.6116}{8.12}
\drawstar{-1.8256}{+0.1444}{9.42}
\drawstar{-1.0788}{+0.0454}{9.35}
\drawstar{-1.4852}{-0.0390}{9.24}
\drawstar{-2.6102}{+0.0180}{8.60}
\drawstar{-2.5614}{+0.0634}{9.33}
\drawstar{-1.5557}{-0.6611}{8.03}
\drawstar{-1.4743}{-1.4000}{6.90}
\drawstar{-1.7429}{-0.0248}{6.20}
\drawstar{-1.4185}{-0.0734}{8.68}
\drawstar{+1.5920}{-1.6494}{9.18}
\drawstar{+2.1185}{-1.7469}{7.97}
\drawstar{+0.9793}{-1.7716}{7.73}
\drawstar{+1.1611}{-2.0939}{4.79}
\drawstar{+0.7301}{-2.1933}{9.15}
\drawstar{+0.7487}{-2.5768}{9.41}
\drawstar{+0.6315}{-2.2199}{9.41}
\drawstar{-0.7363}{-2.5825}{9.29}
\drawstar{-1.3028}{+0.0867}{1.29}
\drawcompass{+1.8000}{-1.8000}{0.1400}
\end{tikzpicture}
