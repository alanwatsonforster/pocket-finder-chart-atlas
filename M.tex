%!TEX root = main.tex

\chapter*{The Messier Objects}
\addcontentsline{toc}{chapter}{\protect\numberline{}The Messier Objects}

The Messier objects are probably the most famous deep-sky objects. I find O’Meara’s \booktitle{Deep-Sky Companion: The Messier Objects} to be excellent  on the origin of the catalog, the appearance of the objects, and their nature.

The following table lists the objects with their J2000 positions (decimal hours of right ascension and degrees of declination), the charts on which they appear in the {\PSA}, their types, and other names. I follow O’Meara in identifying M102 as NGC 5866.

For completeness, I include finder charts for all of the Messier objects, even bright ones like M45.

% I note that page {\itshape xxix} of the {\PSA} gives an index of the charts on which the Messier objects appear, along with their other names.

\clearpage

\begin{table}[t]
\setlength{\tabcolsep}{2pt}
\small
\begin{tabular}{lcclll}
\hline
Name&Position&PSA&Type&Other Names\\
\hline
M1   &$05.6$ $+22$&14&BN &NGC 1952 = Crab Nebula\\
M2   &$21.6$ $-01$&75/77&GC &U81 = NGC 7089\\
M3   &$13.7$ $+28$&43/44&GC &U51 = NGC 5272\\
M4   &$16.4$ $-27$&56/58&GC &U53 = NGC 6121\\
M5   &$15.3$ $+02$&55/57&GC &U52 = NGC 5904\\
M6   &$17.7$ $-32$&58/69&OC &U60 = NGC 6405 = Butterfly Nebula\\
M7   &$17.9$ $-35$&58/69&OC &U62 = NGC 6475\\
M8   &$18.1$ $-24$&67&BN &U64 = NGC 6523/6530 = Lagoon Nebula\\
M9   &$17.3$ $-19$&56&GC &NGC 6333\\
M10  &$17.0$ $-04$&54/56&GC &U57 = NGC 6254\\
M11  &$18.9$ $-06$&65/67&OC &U69 = NGC 6705 = Wild Duck Cluster\\
M12  &$16.8$ $-02$&54/56&GC &U56 = NGC 6218\\
M13  &$16.7$ $+36$&52&GC &U54 = NGC 6205 = Hercules Cluster\\
M14  &$17.6$ $-03$&54&GC &NGC 6402\\
M15  &$21.5$ $+12$&75&GC &U80 = NGC 7078\\
M16  &$18.3$ $-14$&67&BN &NGC 6611 = Eagle Nebula\\
M17  &$18.3$ $-16$&67&BN &U65 = NGC 6618 = Omega Nebula\\
M18  &$18.3$ $-17$&67&OC &NGC 6613\\
M19  &$17.0$ $-26$&56&GC &NGC 6273\\
M20  &$18.0$ $-23$&67&BN &NGC 6514 = Trifid Nebula\\
M21  &$18.1$ $-23$&67&OC &NGC 6531\\
M22  &$18.6$ $-24$&67&GC &U67 = NGC 6656\\
M23  &$17.9$ $-19$&67&OC &NGC 6494\\
M24  &$18.3$ $-19$&67&SC &IC 4715\\
M25  &$18.5$ $-19$&67&OC &IC 4725\\
M26  &$18.8$ $-09$&67&OC &NGC 6694\\
M27  &$20.0$ $+23$&64&PN &U75 = NGC 6853 = Dumbell Nebula\\
M28  &$18.4$ $-25$&67/I&GC &NGC 6626\\
M29  &$20.4$ $+38$&62&OC &NGC 6913\\
M30  &$21.7$ $-23$&77&GC &NGC 7099\\
M31  &$00.7$ $+41$&3&Gal&U3 = NGC 224 = Andromeda Galaxy\\
M32  &$00.7$ $+41$&3&Gal&U2 = NGC 221\\
M33  &$01.6$ $+31$&3&Gal&NGC 598\\
M34  &$02.7$ $+43$&2&OC &NGC 1039\\
M35  &$06.2$ $+24$&14&OC &U27 = NGC 2168\\
M36  &$05.6$ $+34$&12&OC &U23 = NGC 1960\\
M37  &$05.9$ $+33$&12&OC &U26 = NGC 2099\\
M38  &$05.5$ $+36$&12&OC &U22 = NGC 1912\\
M39  &$21.5$ $+48$&73&OC &U82 = NGC 7092\\
M40  &$12.4$ $+58$&32&DS &\\
\hline
\end{tabular}
\end{table}

\begin{table}[t]
\setlength{\tabcolsep}{2pt}
\small
\begin{tabular}{lcclll}
\hline
Name&Position&PSA&Type&Other Names\\
\hline
M41  &$06.8$ $-21$&27&OC &U33 = NGC 2287\\
M42  &$05.6$ $-05$&16/B&BN &U24 = NGC 1976 = Orion Nebula\\
M43  &$05.6$ $-05$&16/B&BN &NGC 1982\\
M44  &$08.7$ $+20$&24&OC &U39 = NGC 2632 = Beehive Cluster\\
M45  &$03.8$ $+24$&15/A&OC &U17 = Mel 22 = Pleiades\\
M46  &$07.7$ $-15$&26&OC &NGC 2437\\
M47  &$07.6$ $-14$&27&OC &NGC 2422\\
M48  &$08.2$ $-06$&26&OC &U38 = NGC 2548\\
M49  &$12.5$ $+08$&45/C&Gal&NGC 4472\\
M50  &$07.0$ $-08$&27&OC &U35 = NGC 2323\\
M51  &$13.5$ $+47$&43&Gal&NGC 5194 = Whirlpool Galaxy\\
M52  &$23.4$ $+62$&72&OC &NGC 7654\\
M53  &$13.2$ $+18$&45&GC &NGC 5024\\
M54  &$18.9$ $-30$&69&GC &NGC 6715\\
M55  &$19.7$ $-31$&68&GC &NGC 6809\\
M56  &$19.3$ $+30$&63&GC &NGC 6779\\
M57  &$18.9$ $+33$&63&PN &U71 = NGC 6720 = Ring Nebula\\
M58  &$12.6$ $+12$&45/C&Gal&NGC 4579\\
M59  &$12.7$ $+12$&45/C&Gal&NGC 4621\\
M60  &$12.7$ $+12$&45/C&Gal&NGC 4649\\
M61  &$12.4$ $+04$&45&Gal&NGC 4303\\
M62  &$17.0$ $-30$&58&GC &U58 = NGC 6266\\
M63  &$13.3$ $+42$&43&Gal&NGC 5055 = Sunflower Galaxy\\
M64  &$12.9$ $+22$&45&Gal&U50 = NGC 4826 = Black-Eye Galaxy\\
M65  &$11.3$ $+13$&34&Gal&NGC 3623\\
M66  &$11.3$ $+13$&34&Gal&NGC 3627\\
M67  &$08.9$ $+12$&24&OC &U40 = NGC 2682\\
M68  &$12.7$ $-27$&47&GC &NGC 4590\\
M69  &$18.5$ $-32$&67&GC &NGC 6637\\
M70  &$18.7$ $-32$&67&GC &NGC 6681\\
M71  &$19.9$ $+19$&64&GC &NGC 6838\\
M72  &$20.9$ $-13$&66/77&GC &NGC 6981\\
M73  &$21.0$ $-13$&66/77&AST&NGC 6994\\
M74  &$01.6$ $+16$&4/5&Gal&NGC 628\\
M75  &$20.1$ $-22$&66&GC &NGC 6864\\
M76  &$01.7$ $+52$&2&PN &NGC 651/650 = Little Dumbell\\
M77  &$02.7$ $-00$&4&Gal&U12 = NGC 1068\\
M78  &$05.8$ $+00$&16&BN &NGC 2068\\
M79  &$05.4$ $-25$&16&GC &NGC 1904\\
M80  &$16.3$ $-23$&56&GC &NGC 6093\\
\hline
\end{tabular}
\end{table}

\begin{table}[t]
\setlength{\tabcolsep}{3pt}
\small
\begin{tabular}{lcclll}
\hline
Name&Position&PSA&Type&Other Names\\
\hline
M81  &$09.9$ $+69$&31&Gal&U41 = NGC 3031\\
M82  &$09.9$ $+70$&31&Gal&U42 = NGC 3034\\
M83  &$13.6$ $-30$&47/48&Gal&NGC 5236\\
M84  &$12.4$ $+13$&45/C&Gal&U45 = NGC 4374\\
M85  &$12.4$ $+18$&45/C&Gal&NGC 4382\\
M86  &$12.4$ $+13$&45/C&Gal&U46 = NGC 4406\\
M87  &$12.5$ $+12$&45/C&Gal&U47 = NGC 4486\\
M88  &$12.5$ $+14$&45/C&Gal&NGC 4501\\
M89  &$12.6$ $+13$&45/C&Gal&NGC 4552\\
M90  &$12.6$ $+13$&45/C&Gal&NGC 4569\\
M91  &$12.6$ $+14$&45/C&Gal&NGC 4548\\
M92  &$17.3$ $+43$&52&GC &U59 = NGC 6341\\
M93  &$07.7$ $-24$&26&OC &NGC 2447\\
M94  &$12.8$ $+41$&43&Gal&U49 = NGC 4736\\
M95  &$10.7$ $+12$&34&Gal&NGC 3351\\
M96  &$10.8$ $+12$&34&Gal&NGC 3368\\
M97  &$11.2$ $+55$&32&PN &NGC 3587\\
M98  &$12.2$ $+15$&45/C&Gal&NGC 4192\\
M99  &$12.3$ $+14$&45/C&Gal&NGC 4254\\
M100 &$12.4$ $+16$&45/C&Gal&NGC 4321\\
M101 &$14.1$ $+54$&42&Gal&NGC 5457\\
M102 &$15.1$ $+56$&42&Gal&NGC 5866\\
M103 &$01.6$ $+61$&3&OC &NGC 581\\
M104 &$12.7$ $-12$&47&Gal&U48 = NGC 4594 = Sombrero Galaxy\\
M105 &$10.8$ $+13$&34&Gal&NGC 3379\\
M106 &$12.3$ $+47$&34&Gal&NGC 4258\\
M107 &$16.5$ $-13$&56&GC &NGC 6171\\
M108 &$11.2$ $+56$&32/33&Gal&NGC 3556\\
M109 &$12.0$ $+53$&32&Gal&NGC 3992\\
M110 &$00.7$ $+42$&3&Gal&NGC 205\\
\hline
\end{tabular}
\end{table}

\clearpage

\twocolumn

\chart{M1}
\chart{M2}
\chart{M3}
\chart{M4}
\chart{M5}
\chart{M6}
\chart{M7}
\chart{M8}
\chart{M9}
\chart{M10}
\chart{M11}
\chart{M12}
\chart{M13}
\chart{M14}
\chart{M15}
\chart{M16}
\chart{M17}
\chart{M18}
\chart{M19}
\chart{M20}
\chart{M21}
\chart{M22}
\chart{M23}
\chart{M24}
\chart{M25}
\chart{M26}
\chart{M27}
\chart{M28}
\chart{M29}
\chart{M30}
\chart{M31}
\chart{M32}
\chart{M33}
\chart{M34}
\chart{M35}
\chart{M36}
\chart{M37}
\chart{M38}
\chart{M39}
\chart{M40}
\chart{M41}
\chart{M42}
\chart{M43}
\chart{M44}
\chart{M45}
\chart{M46}
\chart{M47}
\chart{M48}
\chart{M49}
\chart{M50}
\chart{M51}
\chart{M52}
\chart{M53}
\chart{M54}
\chart{M55}
\chart{M56}
\chart{M57}
\chart{M58}
\chart{M59}
\chart{M60}
\chart{M61}
\chart{M62}
\chart{M63}
\chart{M64}
\chart{M65}
\chart{M66}
\chart{M67}
\chart{M68}
\chart{M69}
\chart{M70}
\chart{M71}
\chart{M72}
\chart{M73}
\chart{M74}
\chart{M75}
\chart{M76}
\chart{M77}
\chart{M78}
\chart{M79}
\chart{M80}
\chart{M81}
\chart{M82}
\chart{M83}
\chart{M84}
\chart{M85}
\chart{M86}
\chart{M87}
\chart{M88}
\chart{M89}
\chart{M90}
\chart{M91}
\chart{M92}
\chart{M93}
\chart{M94}
\chart{M95}
\chart{M96}
\chart{M97}
\chart{M98}
\chart{M99}
\chart{M100}
\chart{M101}
\chart{M102}
\chart{M103}
\chart{M104}
\chart{M105}
\chart{M106}
\chart{M107}
\chart{M108}
\chart{M109}
\chart{M110}

\onecolumn
