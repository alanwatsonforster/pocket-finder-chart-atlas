%!TEX root = main.tex

\chapter{Urban Observing Program}

The Astronomical League Urban Observing Program consists of bright objects selected by Terry Trees for northern observers under light-polluted skies. 

The program consists of 100 objects: 87 deep-sky objects, 12 double stars, and 1 variable star. The deep-sky objects range in declination from $-35$ to $+72$ degrees, cover all seasons, and include 41 Messier objects and 14 Caldwell objects.  

For my convenience, I have labelled the deep-sky objects “U1” to “U87” following the order in right ascension given by Trees, but this is not a standard designation. 

The following table lists the deep-sky objects with their J2000 positions (hours and minutes of right ascension and decimal degrees of declination), the charts on which they appear in the \emph{Pocket Sky Atlas}, their type, and any other names. 

I give finder charts for the deep-sky objects only; the stars are all no fainter than magnitude 5.2 and have Bayer designations, and so a standard all-sky atlas is adequate. For completeness, I include all of the deep-sky objects, even bright ones like the Hyades (U18).

Note that the Hyades (U18) and Coma Star Cluster (U44) are much bigger than the finder charts. They are really binocular objects and better located using a small-scale all-sky atlas. Also note that the $\alpha$ Persei Cluster (U15) and Coma Star Cluster (U44) are not labelled in the {\PSA}. 

For information on the 55 Messier and Caldwell objects, I would refer you to O’Meara’s \booktitle{Deep-Sky Companions} books. His \booktitle{Messier Objects} volume also covers U72 (p.\ 395) and his \booktitle{Caldwell Objects} volume also covers U19 (p.\ 509) and U87 (p.~503). For the remaining objects, you might start at Wikipedia.

\clearpage

\begin{table}[p]
\setlength{\tabcolsep}{3pt}
\scriptsize
\begin{tabular}{llllll}
\hline
Name&Position&PSA&Type&Other Names\\
\hline
U1&$00$ $30$ $+60.2$&1/3/72&OC&NGC 129\\
U2&$00$ $43$ $+40.9$&3/72&Gal&M32 / NGC 221\\
U3&$00$ $43$ $+41.3$&3/72&Gal&M31 / NGC 224 / Andromeda Galaxy\\
U4&$01$ $20$ $+58.3$&1/3/72&OC&C13 / NGC 457 / Owl Cluster\\
U5&$01$ $46$ $+61.2$&1/2&OC&C10 / NGC 663\\
U6&$01$ $48$ $+72.0$&1/11&OC&Cr 463\\
U7&$01$ $58$ $+37.8$&2&OC&C28 / NGC 752\\
U8&$02$ $15$ $+59.3$&1/2/13&OC&Stock 2\\
U9&$02$ $19$ $+57.1$&1/2/13&OC&C14 / NGC 869 / h Persei / Double Cluster\\
U10&$02$ $22$ $+57.1$&1/2/13&OC&C14 / NGC 884 / $\chi$ Persei / Double Cluster\\
U11&$02$ $37$ $+56.0$&1/2/13&OC&Tr 2\\
U12&$02$ $43$ $-00.0$&4/6&Gal&M77 / NGC 1068 / Squid Galaxy\\
U13&$03$ $12$ $+63.3$&1/2/11/13&OC&Tr 3\\
U14&$03$ $16$ $+60.0$&2/11/13&OC&Stock 23 / Pazmino's Cluster\\
U15&$03$ $27$ $+48.8$&2/13&OC&Mel 20 / Cr 15 / $\alpha$ Persei Cluster\\
U16&$03$ $32$ $+37.4$&2/13&OC&NGC 1342\\
U17&$03$ $47$ $+24.1$&13/15/A&OC&M45 / Mel 22 / Cr 42 / Pleiades\\
U18&$04$ $27$ $+15.9$&15&OC&C41 / Mel 25 / Cr 50 / Hyades\\
U19&$04$ $46$ $+19.1$&14/15&OC&NGC 1647\\
U20&$05$ $11$ $+16.5$&14&OC&NGC 1807\\
U21&$05$ $12$ $+16.7$&14&OC&NGC 1817\\
U22&$05$ $29$ $+35.9$&12/14&OC&M38 / NGC 1912 / Starfish Cluster\\
U23&$05$ $36$ $+34.1$&12/14&OC&M36 / NGC 1960\\
U24&$05$ $35$ $-05.4$&16/B&BN&M42 / NGC 1976 / Orion Nebula\\
U25&$05$ $35$ $-04.4$&16/B&OC&NGC 1981\\
U26&$05$ $52$ $+32.6$&12/14/23/25&OC&M37 / NGC 2099\\
U27&$06$ $09$ $+24.4$&12/14/23/25&OC&M35 / NGC 2168\\
U28&$06$ $08$ $+14.0$&14/25&OC&NGC 2169\\
U29&$06$ $27$ $-04.8$&25/27&OC&NGC 2232\\
U30&$06$ $32$ $+04.9$&25/E&OC&C50 / NGC 2244  / Satellite Cluster\\
U31&$06$ $41$ $+09.9$&25/E&OC&NGC 2264 / Christmas Tree Cluster\\
U32&$06$ $48$ $+41.1$&12/23&OC&NGC 2281\\
U33&$06$ $46$ $-20.8$&27&OC&M41 / NGC 2287 / Little Beehive Cluster\\
U34&$06$ $52$ $+00.5$&25/27&OC&NGC 2301\\
U35&$07$ $03$ $-08.4$&27&OC&M50 / NGC 2323\\
U36&$07$ $29$ $+20.9$&25&PN&C39 / NGC 2392 / Eskimo Nebula\\
U37&$08$ $11$ $-12.8$&26&OC&NGC 2539\\
U38&$08$ $14$ $-05.8$&26&OC&M48 / NGC 2548\\
U39&$08$ $40$ $+19.7$&24/35&OC&M44 / NGC 2632 / Beehive Cluster\\
U40&$08$ $51$ $+11.8$&24/35&OC&M67 / NGC 2682\\
U41&$09$ $56$ $+69.1$&21/31&Gal&M81 / NGC 3031 / Bode’s Galaxy\\
U42&$09$ $56$ $+69.7$&21/31&Gal&M82 / NGC 3034 / Cigar Galaxy\\
U43&$10$ $25$ $-18.6$&36/37&PN&C59 / NGC 3242 / Ghost of Jupiter\\
U44&$12$ $26$ $+25.9$&45&OC&Mel 111 / Cr 256 / Coma Star Cluster\\
U45&$12$ $25$ $+12.9$&45/C&Gal&M84 / NGC 4374\\
U46&$12$ $26$ $+12.9$&45/C&Gal&M86 / NGC 4406\\
U47&$12$ $31$ $+12.4$&45/C&Gal&M87 / NGC 4486\\
U48&$12$ $40$ $-11.6$&47&Gal&M104 / NGC 4594 / Sombrero Galaxy\\
U49&$12$ $51$ $+41.1$&32/43&Gal&M94 / NGC 4736\\
\hline
\end{tabular}
\end{table}
\clearpage

\begin{table}[p]
\setlength{\tabcolsep}{3pt}
\scriptsize
\begin{tabular}{llllll}
\hline
Name&Position&PSA&Type&Other Names\\
\hline
U50&$12$ $57$ $+21.7$&45&Gal&M64 / NGC 4826 / Black-Eye Galaxy\\
U51&$13$ $42$ $+28.4$&43/44&GC&M3 / NGC 5272\\
U52&$15$ $19$ $+02.1$&55/57&GC&M5 / NGC 5904\\
U53&$16$ $24$ $-26.5$&56/58&GC&M4 / NGC 6121\\
U54&$16$ $42$ $+36.5$&52&GC&M13 / NGC 6205 / Hercules Globular Cluster\\
U55&$16$ $44$ $+23.8$&52/54&PN&NGC 6210\\
U56&$16$ $47$ $-01.9$&54/56&GC&M12 / NGC 6218\\
U57&$16$ $57$ $-04.1$&54/56&GC&M10 / NGC 6254\\
U58&$17$ $01$ $-30.1$&56/58&GC&M62 / NGC 6266\\
U59&$17$ $17$ $+43.1$&52/63&GC&M92 / NGC 6341\\
U60&$17$ $40$ $-32.3$&56/58/66/68/J&OC&M6 / NGC 6405 / Butterfly Cluster\\
U61&$17$ $46$ $+05.7$&54/65&OC&IC 4665\\
U62&$17$ $54$ $-34.8$&58/67/69/J&OC&M7 / NGC 6475\\
U63&$18$ $03$ $-27.9$&67/69/I&OC&NGC 6520\\
U64&$18$ $04$ $-24.4$&67/69/I&BN&M8 / NGC 6523 / Lagoon Nebula\\
U65&$18$ $21$ $-16.2$&67/I&BN&M17 / NGC 6618 / Omega Nebula\\
U66&$18$ $27$ $+06.5$&65&OC&NGC 6633 / Tweedledum Cluster\\
U67&$18$ $36$ $-23.9$&67/69/I&GC&M22 / NGC 6656\\
U68&$18$ $39$ $+05.4$&65&OC&IC 4756 / Tweedledee Cluster\\
U69&$18$ $51$ $-06.3$&65/67&OC&M11 / NGC 6705 / Wild Duck Cluster\\
U70&$18$ $52$ $+10.3$&65&OC&NGC 6709\\
U71&$18$ $54$ $+33.0$&63/65&PN&M57 / NGC 6720 / Ring Nebula\\
U72&$19$ $25$ $+20.2$&64/65&OC&Cr 399 / Brocchi's Cluster / Coathanger Cluster\\
U73&$19$ $44$ $-14.2$&66&PN&NGC 6818 / Little Gem Nebula\\
U74&$19$ $45$ $+50.5$&62/73&PN&C15 / NGC 6826 / Blinking Planetary\\
U75&$20$ $00$ $+22.7$&62/64&PN&M27 / NGC 6853 / Dumbell Nebula\\
U76&$20$ $23$ $+40.8$&62/73&OC&NGC 6910\\
U77&$20$ $34$ $+07.4$&64&GC&C47 / NGC 6934\\
U78&$20$ $35$ $+28.3$&62/64/73&OC&NGC 6940\\
U79&$21$ $04$ $-11.4$&77&PN&C55 / NGC 7009 / Saturn Nebula\\
U80&$21$ $30$ $+12.2$&75&GC&M15 / NGC 7078\\
U81&$21$ $33$ $-00.8$&75/77&GC&M2 / NGC 7089\\
U82&$21$ $32$ $+48.4$&62/73&OC&M39 / NGC 7092\\
U83&$21$ $54$ $+62.6$&71/73&OC&NGC 7160\\
U84&$22$ $05$ $+46.5$&62/73&OC&NGC 7209\\
U85&$22$ $15$ $+49.9$&62/73&OC&C16 / NGC 7243\\
U86&$23$ $26$ $+42.5$&3/72&PN&C22 / NGC 7662 / Blue Snowball Nebula\\
U87&$23$ $57$ $+56.7$&3/71/72&OC&NGC 7789 / Caroline’s Rose\\
\hline
\end{tabular}
\end{table}

\clearpage

\twocolumn

\chart{U1}
\chart{U2}
\chart{U3}
\chart{U4}
\chart{U5}
\chart{U6}
\chart{U7}
\chart{U8}
\chart{U9}
\chart{U10}
\chart{U11}
\chart{U12}
\chart{U13}
\chart{U14}
\chart{U15}
\chart{U16}
\chart{U17}
\chart{U18}
\chart{U19}
\chart{U20}
\chart{U21}
\chart{U22}
\chart{U23}
\chart{U24}
\chart{U25}
\chart{U26}
\chart{U27}
\chart{U28}
\chart{U29}
\chart{U30}
\chart{U31}
\chart{U32}
\chart{U33}
\chart{U34}
\chart{U35}
\chart{U36}
\chart{U37}
\chart{U38}
\chart{U39}
\chart{U40}
\chart{U41}
\chart{U42}
\chart{U43}
\chart{U44}
\chart{U45}
\chart{U46}
\chart{U47}
\chart{U48}
\chart{U49}
\chart{U50}
\chart{U51}
\chart{U52}
\chart{U53}
\chart{U54}
\chart{U55}
\chart{U56}
\chart{U57}
\chart{U58}
\chart{U59}
\chart{U60}
\chart{U61}
\chart{U62}
\chart{U63}
\chart{U64}
\chart{U65}
\chart{U66}
\chart{U67}
\chart{U68}
\chart{U69}
\chart{U70}
\chart{U71}
\chart{U72}
\chart{U73}
\chart{U74}
\chart{U75}
\chart{U76}
\chart{U77}
\chart{U78}
\chart{U79}
\chart{U80}
\chart{U81}
\chart{U82}
\chart{U83}
\chart{U84}
\chart{U85}
\chart{U86}
\chart{U87}

\onecolumn
